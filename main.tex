\documentclass[10pt, conference]{IEEEtran}
% \IEEEoverridecommandlockouts
% The preceding line is only needed to identify funding in the first footnote. If that is unneeded, please comment it out.
\usepackage{cite}
\usepackage{amsmath,amssymb,amsfonts}
\usepackage{algorithmic}
\usepackage{graphicx}
\usepackage{textcomp}
\usepackage{xcolor}
\usepackage{acronym}
\def\BibTeX{{\rm B\kern-.05em{\sc i\kern-.025em b}\kern-.08em
    T\kern-.1667em\lower.7ex\hbox{E}\kern-.125emX}}

\input{preamble}
% !TEX root = main.tex

\acrodef{APK}{Android Application Package}
\acrodef{AI}{Artificial Intelligence}
\acrodef{WCAG}{Web Content Accessibility Guidelines}

\begin{document}

\title{Autonomous Agents for Accessibility: Simulating Visual Impairments in Web Interfaces}

\author{\IEEEauthorblockN{1\textsuperscript{st} Juan Diego Yepes-Parra}
\IEEEauthorblockA{\textit{Universidad de los Andes} \\
Bogotá, Colombia \\
j.yepes@uniandes.edu.co}
\and
\IEEEauthorblockN{2\textsuperscript{nd} Camilo Andrés Escobar-Velásquez}
\IEEEauthorblockA{\textit{Universidad de los Andes} \\
Bogotá, Colombia \\
ca.escobar2434@uniandes.edu.co}
}

\maketitle

\begin{abstract}
Static code analysis cannot detect real-time interaction issues faced by users with disabilities. We introduce a multimodal AI agent framework that simulates user interactions under visual impairments without code access. These agents are exposed to perceptual filters representing conditions such as glaucoma and myopia, and receive both visual input and audio output generated by screen readers. We outline a framework to evaluate how such filters affect user behavior, task success, and interface usability. Our approach aims to uncover visual accessibility flaws that emerge only through impaired perception. We also reflect on the ethical considerations of simulating user experiences and outline research directions toward agent-based, perception-driven accessibility evaluation.
\end{abstract}

\begin{IEEEkeywords}
Web Accessibility; Autonomous \ac{AI} Agents; Automated Testing
\end{IEEEkeywords}


% !TEX root = main.tex

\section{Introduction}

Web accessibility is central to inclusive software design, yet existing evaluation methods predominantly focus on code conformance, such as \ac{WCAG} rule checks, often overlooking how individuals with disabilities actually navigate and perceive interfaces in practice \cite{ara2024inclusive}. Although these tools are useful for guidance and a first approach, they can be unable to thoroughly assess how these impairments affect real-time usability, task completion, or perception of interface elements. While often ignoring behavioral context; they do not simulate navigation, focus order, or sequential interaction flows that can significantly impact accessibility. Individuals with disabilities have diverse needs and experiences, many of which may not be fully addressed by level-A \ac{WCAG} guidelines alone.

Although the access to information is a human right, the urgency of dynamic and human-centered accessibility evaluation is also supported by data. Approximately 2.2 billion people globally have some sort of visual impairment, including both near and distant vision issues \cite{who2023vision}. In Colombia, among the estimated 2.65 million people living with a disability by the DANE, approximately 57\% report that vision-related activities present the greatest challenges\cite{DANE2022}. In bigger countries like the U.S.A., the C.D.C. reports that, about 5.5\% of adults (nearly 19 million people) have blindness or serious difficulty seeing\cite{cdc2025disabilities}. Furthermore, screen reader users find that the most problematic items to interact with in a webpage are CAPTCHA, interactive elements, ambiguous links or buttons, unexpected screen changes, lack of keyboard support, among others\cite{webaimsurvey2025}. Even more tellingly, these users extract information from data visualizations 61\% less accurately and take 211\% more time compared to sighted users \cite{wobbrock2021assets}.

We explore the possibility of approaching accessibility testing using autonomous \ac{AI} agents capable of interacting with web pages visually while being exposed to simulated visual impairments. These agents are prompted with specific user tasks (e.g., locating a button, submitting a form) and attempt to complete them. The agents can also integrate outputs from assistive technologies, for example, capturing a screen reader's textual narration as an additional input channel. 

This multi-modal input allows the agent to simulate how a user with various levels of visual impairment (such as glaucoma, cataracts, myopia or low vision, among others) interacts with content, opening the door for dynamic automated testing other existing tools might be unable to uncover. For instance, detecting unuseful alt text or mislabeled controls or mismatches between rendered content and screen reader output, thereby uncovering potential issues that visual or static code checks miss. 

This paper outlines our motivation, proposed approach, poses future evaluation research questions, and discusses implementation considerations for such a system.

% !TEX root = main.tex

\section{Related Work}
\subsection{Web accessibility compliance}

Conventional web accessibility assessments primarily rely on static checkers that verify compliance with \ac{WCAG} criteria by inspecting HTML and CSS structures. Tools including WAVE by WebAIM\cite{webaim_wave_2025} and IBM's NPM accessibility-checker\cite{ibm_accessibility_checker_2025} exemplify this approach. However, several studies have revealed significant limitations in these tools, noting they often lack semantic awareness and fail to consider user perspectives, leading to incomplete or misleading results \cite{ara2024inclusive}. 

For instance, Todorov et al. evaluated Bulgarian museum websites uncovered widespread accessibility failures despite technical correctness \cite{todorov2022accessibility}. Similarly, Inal et al. conducted a study of Norwegian municipal websites and found that although legislation mandated \ac{WCAG} compliance, issues such as low-quality alternative text persisted.
Chiou et al. further observed that many common web accessibility issues arise in responsive sites during resizing, highlighting the importance of evaluating accessibility across different device viewports\cite{chiou2024automatically}.

Complimentarily, \ac{WCAG} conformance varies by level, from A (lower) to AAA (higher), meaning that a website labeled as “compliant” may still fall short of more stringent and specific accessibility standards. These findings underscore the need for real-time evaluations that reflect how visually impaired users interact with web interfaces. 
While automated tools are useful for early feedback, they are unable to identify all critical accessibility, functionality, and usability issues affecting this population \cite{todorov2022accessibility}.

\subsection{Automated accessibility Testing}

Recent studies introduce hybrid frameworks that combine guidelines with heuristic, automated, or AI-driven methods to improve accessibility evaluation. Watanabe et al.~\cite{watanabe2024accessibility} show that machine learning can detect ARIA landmarks in web apps, revealing how classifiers may infer structure when key accessibility tags are missing. Similarly, models have been used to identify and remediate accessibility issues, helping sites align with accessibility standards. Some approaches analyze source code to suggest fixes~\cite{ramineni2024leveraging, kuszczynski2023comparative}, while others operate on rendered pages using prompt-based methods~\cite{he2025enhancing}. Mehralian et al.~\cite{mehralian2025automated} propose a rule-based system for testing dynamic content in Android apps, and Tafreshipour et al.~\cite{tafreshipour2024ma11y} show that mutation testing can uncover additional errors by exploring app states and applying accessibility rules. These efforts reflect growing recognition of the limits of static, rule-based tools, while still relying on source code analysis, not simulating user behavior.


\subsection{Autonomous AI agents}

The use of autonomous \ac{AI} agents to simulate user interaction with web interfaces has also been explored. Lu et al. \cite{lu2025uxagent} introduce UXAgent; a notable system that uses LLM agents to mimic thousands of diverse user personas in web usability studies. The agents interact with live websites via browser automation, providing qualitative and quantitative feedback that supports iterative UX design. 

Complementing this, a GitHub Copilot extension that proactively embeds accessibility guidance into the coding workflow was unveiled by Mowar et al. \cite{mowar2025codea11y}. Their study shows how \ac{AI} assistants can suggest accessible \ac{UI} code, highlight missing attributes, and prompt manual verification during development.

While not strictly agent-based, Zhong et al. \cite{zhong2025screenaudit} leveraged large language models to identify discrepancies between screen reader outputs, providing a novel approach to detecting accessibility errors.

Another recent advancement in this area is AXNav by Taeb et al. \cite{taeb2024axnav}, a system that interprets mobile accessibility test instructions written in natural language and executes them on remote cloud devices using an LLM-based multiagent planner. This approach demonstrates how LLM-driven agents can automate complex evaluations and provide actionable, context-rich feedback for developers.

Multimodal agents that adaptively present content based on user needs, transforming visual content into speech or simplified visuals for users with auditory or visual processing disorders were also proposed by Rajagopal et al. \cite{rajagopal2023design}. While not focused on web testing, their work shows how agents can model disability-specific interactions across modalities.

\section{Automated Testing} % ? No se si dejar este titulo, y si si seria otra seccion aparte

Dynamic testing can help developers and stakeholders verify and validate that running software is working as expected\cite{vasquez2018continuous}. In this case, automated testing is the enabler for faster, more reliable and overall optimized testing. Automated testing leveraged by machine learning and \ac{AI} are increasingly becoming more popular and sophisticated, therefore using them in this scenario is a big step forward for ensuring accessibility.

Intelligent agents can interact with web content without access to the underlying code, relying instead on the same content the final user is interacting with\cite{lanham2025ai, wang2024survey, lu2025uxagent}. They can also be adapted based on feedback, and can process rich multimodal inputs, namely screenshots and screen reader text, enabling them to reason about both the visual layout and the spoken feedback of the user interface simultaneously. They are also able to use different interaction techniques, like keyboard only, mouse only, etc. 

Finally, LLM agents excel at open-ended reasoning and can provide qualitative insights, alongside quantitative logs. However, they require careful prompting and can be slower or less predictable.
\vspace{-8pt}
% !TEX root = main.tex

\section{System design}
\subsection{Visual Filtering}

Simulating visual impairments through perceptual filters allows us to approximate the experiences of users with vision loss conditions. These simulations (see figures. \ref{fig:glaucoma-filters} and \ref{fig:myopia-filters}) were implemented using the Visual Impairments Simulator Chrome extension \cite{visual_impairments_simulator}. These tools are inherently limited; since visual disabilities are highly individualized, and no filter can perfectly replicate the subjective visual experience of every user.

\begin{figure}
    \centering
    \includegraphics[width=115pt]{imgs/glaucoma-filter.png}
    \includegraphics[width=115pt]{imgs/no-glaucoma-filter.png}
    \caption{Left: Glaucoma filter applied to an example form requiring a signature. The "Submit" button is no longer visible, making it difficult to locate. Right: Original version.}
    \vspace{-13pt}
    \label{fig:glaucoma-filters}
\end{figure}

\begin{figure}
    \centering
    \includegraphics[width=115pt]{imgs/myopia-filter.png}
    \includegraphics[width=115pt]{imgs/no-myopia-filter.png}
    \caption{Left: Myopia filter (-3 diopters) applied to a social media webpage, reducing clarity and edge sharpness. Right: Original version.}
    \vspace{-13pt}
    \label{fig:myopia-filters}
\end{figure}

% * The paper lacks an analysis of the target population and the specific visual impairments considered. The authors do not clearly articulate which types of visual impairments are modeled, their characteristics, or whether the chosen visual filtering techniques can adequately represent their impact on Web browsing.

In light of these considerations, the beta testing of this agent will focus on a defined set of visual impairments. The prototype includes simulations of glaucoma (peripheral field loss and tunnel vision\cite{Cassel2021EyeBook}), diabetic retinopathy (scattered blind spots and patchy visual fields\cite{Cassel2021EyeBook}), cataracts (global blurriness and reduced contrast sensitivity\cite{Cassel2021EyeBook}), and myopia (blur at distance with varying levels of diopter correction\cite{Cassel2021EyeBook}). These conditions were selected as representative examples because they encompass distinct perceptual characteristics, in ways that reflect known physiological effects.

In addition, experimenting with different viewports, font sizes, and other browser configurations\cite{chiou2024automatically} will be done to test how these factors affect usability.

Future work involves collaboration with ophthalmologists and vision science experts to improve the clinical accuracy of these filters. Their insight can guide the calibration of filter parameters and help us design simulations that more closely resemble the lived experiences of users.

\subsection{Assistive output integration}

\begin{figure}
    \centering
    \includegraphics[width=1\linewidth]{imgs/flow.png}
    \caption{Proposed pipeline for the agent feeding inputs}
    \vspace{-13pt}
\label{fig:pipeline}
\end{figure}

The agent must capture what an assistive technology user would hear and do. For screen reader output, we can integrate a driver API, such as Guidepup\cite{guidepup2025} that programmatically controls VoiceOver\cite{voiceover2024} or NVDA\cite{nvda2024}.
This driver issues the same keyboard commands a user would and exposes the spoken utterances\cite{guidepup2025}. The agent framework would invoke this API at each step and record the resulting string of text (including element role, name, state) as sensory input.

% * Explain how issues are detected from agent logs — show concrete log features, detection rules/heuristics or classifiers, and example mappings (log → issue).

For navigation logs, the agent can log focus events. As the agent presses Tab, arrow keys, click, Enter, etc., a browser automation framework, namely Playwright\cite{playwright2025}, Selenium\cite{garcia2024selenium}, or Kraken\cite{ravelo2023kraken}, can attach listeners to record each focus transition and action. This yields a sequence of elements visited in order\cite{ravelo2023kraken}. 

The resulting logs constitute a detailed record of the agent's navigation path, enabling systematic analysis of interaction patterns. These logs can be examined to identify instances where interface elements were bypassed, or to detect deviations from the expected focus order. Furthermore, replaying recorded sessions facilitates a nuanced assessment of element discoverability, the interpretation of screen reader output, and the factors influencing task completion or failure. Together, these multimodal data streams comprehensively characterize the agent's perceptual state at each interaction step. 

% * metadata, model-based representation of what the user does. click and all interactions also is enriched by ARIA labels

The agent can also log accessibility metadata, thereby having the ability to cross-check perception. This means reading the ARIA attributes for the current focused element. For instance, when an element is in focus, the agent can query its ARIA label, via the \ac{AOM} or other APIs. This is then compared to the other information (visual, audio). If the visible text of a button says “Search” but its ARIA label is “Submit” that mismatch is noteworthy. Altogether, ARIA attributes and accessibility metadata enrich the logs by providing a semantic model of user actions and interface states, allowing for more precise analysis and how accessibility features are utilized or bypassed. This is useful for verifying consistency, which is one of the guidelines on \ac{WCAG} and IBM's\cite{ibm2025accessibility}. 


\subsection{Decision and Action Module}


Our approach consists of prompting the agent with a basic user goal. E.g. “given this screenshot and screen reader transcript, where is the 'Submit' button?”. The agent will then attempt to complete the task, using the screen reader and seeing through the visual impairment filter applied to the~interface.
Consider one scenario which may involve an \ac{AI} agent attempting to find and click a "Submit" button after filling out a form. With a glaucoma filter applied (see Fig. \ref{fig:glaucoma-filters}), the button may become difficult to distinguish due to peripheral blur or low contrast. The agent has to decide which interaction method works best (pointer, keyboard, etc) and execute the decision.

\subsection{Workflow Definition}

Each testing task needs to be defined in advance. This can be done either with a script or using natural language to define a goal. Then, the agent will execute said task in a closed loop that has a completion or failure condition. All in all, the loop can be structured in this way (see figure \ref{fig:pipeline}).

\underline{Steps:} Perception (capture current state: screenshot, screen-reader output, optionally \ac{AOM} information) $\rightarrow$ Decision (agent chooses next action, e.g., LLM generates plan and writes a rule-based script that framework follows) $\rightarrow$ Action (execute step) $\rightarrow$ Measure (log outputs, check goal, repeat).


\subsection{Metrics and Analysis}

After execution is complete, some of the metrics we propose that the system displays are both classic usability and accessibility-specific. These include the agent's task success rate, efficiency (measured by completion time or number of actions), and the frequency of interaction errors, missed clicks or incorrect actions that require backtracking. 

We also include the consistency between screen reader output and the visible \ac{UI}, flagging any mismatches between spoken labels and on-screen text or roles. Visual robustness can also be examined by analyzing the webpage before the filter, discovering layout faults like overlapping or off-screen elements, among others. 

Heuristic checks are performed during testing, among them are verifying that text scales appropriately in high-contrast or large-text modes and monitoring for navigation loops. Throughout the process, we aggregate logs of the agent's actions and screen reader output for offline analysis, enabling manual review or further explanation by the agent. By comparing these metrics across different simulated conditions, we can quantify the impact of visual impairments on usability and identify both layout and semantic accessibility issues.


\section{Evaluation}

To investigate the feasibility and effectiveness of using autonomous AI agents for accessibility evaluation, we pose several research questions. We ask whether autonomous AI agents can realistically emulate the interaction patterns of users with vision-related impairments when navigating web interfaces. This includes examining which visual filters or perceptual constraints most effectively represent different types of visual impairments in a simulated context, and what design principles are necessary for agents to approximate visual impairments through perception-based (rather than code-aware) interaction.

We also explore whether these agents can surface accessibility problems that static tools overlook, like poor contrast visibility, confusing focus order, or dynamic content that is not screen-reader friendly, to name a few. Another important question is whether agent-based testing, when combined with simulated impairments, can produce accessibility assessments that are reliable and generalizable. We are interested in the extent to which these systems can internalize accessibility heuristics through observation of human interaction data, rather than relying solely on static rule sets.

Finally, we investigate how combining simulated visual impairment filters with screen reader output affects the agent's ability to detect accessibility issues, and whether multi-modal input can uncover problems, say, missing alt text, that would not surface if using only visual filtering.

To evaluate our approach, we will conduct experiments on a curated set of benchmark web pages, including public sites with documented accessibility issues. Inspired by Alameer et al.\cite{alameer2016detecting}, who compiled websites with internationalization challenges, we may extend our benchmarks to include similar cases. For each page, we will define representative tasks, and assess agent performance across different visual filters and input modalities. We plan to complement these quantitative results with qualitative feedback from accessibility experts and targeted user studies, to further validate the effectiveness and generalizability of agent-based accessibility testing.

\vspace{-4pt}
% !TEX root = main.tex

\section{Conclusion \& Future Work}

This paper proposes a new direction for accessibility evaluation: the use of autonomous, multi-modal \ac{AI} agents that simulate visual impairments and interact with web content through perception, not code. By combining visual filters, screen reader output, and task-based prompting, this approach aims to approximate the experiences of users with diverse visual disabilities and to surface accessibility issues that may be missed by static tools.

The potential of this method lies in its ability to reveal failures that only emerge under perceptual constraints, and to provide richer, multi-modal insights into web accessibility. At the same time, we recognize the methodological challenges of using simulated agents as proxies for real users, and stress the importance of human validation.

Future work will involve comparing this approach with existing methods, such as seeking consultation services from accessibility experts, comparison with real user evaluations and static methods. Together with a detailed analysis of system performance and cost-effectiveness, weighing the benefits of automation against the increased computational demands. Ultimately, we hope this research will lay the groundwork for more dynamic, user-centered accessibility evaluation methods and inspire further research in this area.


% * The paper could be improved by introducing further analysis and discussion on the system overhead, as deploying compute-intensive LLM-based AI agents requires much more resources than conventional static approaches.

% * future work: the impact and cost-benefit of using this instead of static, carlos example, cost of hiring someone v.s. using this solution. e.g. inference


% hay que hacer esto?
% \section*{Acknowledgment} 


\balance
\bibliographystyle{IEEEtran}
\bibliography{local,bib/testing,bib/tools}

\end{document}
