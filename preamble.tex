
%---- PACKAGES
%\usepackage{todonotes}
\usepackage{amssymb}
\usepackage{hyperref}
\usepackage[plain]{fancyref}
\usepackage{ifdraft}
%\usepackage{subcaption}
\let\labelindent\relax
\usepackage[inline]{enumitem}
\usepackage{xcolor}
\usepackage{xspace}
\usepackage[final]{listings}
\usepackage{acronym}
\usepackage{url}
\usepackage{amsmath}
\usepackage{amssymb}
\usepackage{booktabs} % For formal tables
\usepackage{subfig}
\usepackage{balance}
\usepackage{dirtree}


\usepackage[ruled]{algorithm2e} % For algorithms
\renewcommand{\algorithmcfname}{ALGORITHM}
\newcommand{\subparagraph}{}


\usepackage{etoolbox}
\makeatletter
\patchcmd{\@makecaption}
  {\scshape}
  {}
  {}
  {}
\patchcmd{\@makecaption}
  {\\}
  {.\ }
  {}
  {}
\makeatother

%\let\refsection\relax
%\usepackage
%  [backend=bibtex,
%  style=ieee,
%   maxnames=5,
%   firstinits=true,
%   hyperref=true,
%   natbib=true,
%   url=false,
%   doi=false]{biblatex}



%
%----[ Biber ]----
%\addbibresource[datatype=bibtex]{local.bib}
%\addbibresource[datatype=bibtex]{bib/tools.bib}
%\addbibresource[datatype=bibtex]{bib/compsci.bib}
%\addbibresource[datatype=bibtex]{bib/learning.bib}
%\nocite{*}
%
%\newbibmacro{name:newformat}{%
%    \printnames{authors}
%%   \textbf{\namepartfamily}  % #1->\namepartfamily, #2->\namepartfamilyi
%%   \textbf{\namepartgiven}   % #3->\namepartgiven,  #4->\namepartgiveni
%%   [prefix: \namepartprefix] % #5->\namepartprefix, #6->\namepartprefixi
%%   [suffix: \namepartsuffix] % #7->\namepartsuffix, #8->\namepartsuffixi
%}
%
%\DeclareNameFormat{newformat}{%
%  \usebibmacro{name:newformat}{\textbf{#1}}{\textbf{#3}}{#5}{#7}%
%  \usebibmacro{name:andothers}%
%%  \nameparts{#1}% split the name data, will not be necessary in future versions
%%  \usebibmacro{name:newformat}%
%%  \usebibmacro{name:andothers}%
%}
%

%\AtEveryBibitem
%{
%   \clearlist{address}
%   \clearfield{date}
%   \clearfield{doi}
%   \clearfield{eprint}
%   \clearfield{isbn}
%   \clearfield{issn}
%   \clearfield{month}
%   \clearfield{note}
%%   \clearfield{pages}
%   \clearlist{location}
%%   \clearfield{series}
%   \clearfield{url}
%   \clearname{editor}
%   \ifentrytype{inproceedings}
%     {\clearfield{day}
%      \clearfield{month}
%      \clearfield{volume}}{}
%}
%
%\DeclareFieldFormat*{title}{\textsl{#1}\isdot}
%\DeclareFieldFormat*{journaltitle}{#1}
%\DeclareFieldFormat*{booktitle}{#1}
%
%\renewbibmacro{in:}{} % supress 'In: ' form
%
%\DeclareSourcemap
% {\maps[datatype=bibtex,overwrite]
%   {% Tag entries (through keywords)
%    \map
%      {\step[fieldsource=booktitle,
%       match=\regexp{[Pp]roceedings}, replace={Proc.}]}
%        \map
%      {\step[fieldsource=booktitle,
%       match=\regexp{[Ii]nternational\s+[Cc]onference}, replace={Intl. Conf.}]}
%    \map
%      {\step[fieldsource=journal,
%       match=\regexp{[Jj]ournal}, replace={Jour.}]}
%    \map
%      {\step[fieldsource=journal,
%       match=\regexp{[Tt]ransactions}, replace={Trans.}]}
%    \map
%      {\step[fieldsource=booktitle,
%       match=\regexp{[Pp]roceedings\s+of\s+the.+[Ee]uropean\s+[Cc]onference\s+in}, replace={European Conf. in}]}
%    \map
%      {\step[fieldsource=booktitle,
%       match=\regexp{In\s+[Pp]roceedings\s+of\s+the.+[Ss]ymposium\s+on}, replace={Symp. on}]}
%    \map
%      {\step[fieldsource=booktitle,
%       match=\regexp{[Pp]roceedings\s+of\s+the.+[Ii]nternational\s+[Cc]onference\s+on}, replace={Intl. Conf. on}]}
%    \map
%      {\step[fieldsource=booktitle,
%       match=\regexp{[Pp]roceedings\s+of\s+the.+[Ii]nternational\s+[Ww]orkshop\s+on}, replace={Intl. Workshop on}]}}}
%

%color
\definecolor{OliveGreen}{rgb}{0,0.6,0.3}

%References
%% Listings
\def\fref{\Fref} % treat all \frefs as \Frefs
\renewcommand{\lstlistingname}{Snippet}
\newcommand*{\fancyreflstlabelprefix}{lst}
\newcommand*{\Freflstname}{\lstlistingname}
\newcommand*{\freflstname}{\MakeLowercase{\lstlistingname}}
\Frefformat{vario}{\fancyreflstlabelprefix}%
  {\Freflstname\fancyrefdefaultspacing#1#3}
\frefformat{vario}{\fancyreflstlabelprefix}%
  {\freflstname\fancyrefdefaultspacing#1#3}
\Frefformat{plain}{\fancyreflstlabelprefix}%
  {\Freflstname\fancyrefdefaultspacing#1}
\frefformat{plain}{\fancyreflstlabelprefix}%
  {\freflstname\fancyrefdefaultspacing#1}

\renewcommand{\tablename}{Table}  
  
% ln delimiter
\newcommand*{\fancyreflnlabelprefix}{ln}
\newcommand*{\Freflnname}{Line}
\newcommand*{\freflnname}{\MakeLowercase{\Freflnname}}
\Frefformat{vario}{\fancyreflnlabelprefix}%
  {\Freflnname\fancyrefdefaultspacing#1#3}
\frefformat{vario}{\fancyreflnlabelprefix}%
  {\freflnname\fancyrefdefaultspacing#1#3}
\Frefformat{plain}{\fancyreflnlabelprefix}%
  {\Freflnname\fancyrefdefaultspacing#1}
\frefformat{plain}{\fancyreflnlabelprefix}%
  {\freflnname\fancyrefdefaultspacing#1}    


%JavaScript definition
\lstdefinelanguage{JavaScript}{
keywords={typeof, new, true, false, catch, function, return, null, catch, switch, var, if, in, for, while, do, else, case, break, throw, this, instanceof},
keywordstyle=\color{purple}\bfseries,
ndkeywords={},
ndkeywordstyle=\color{blue}\bfseries,
identifierstyle=\color{black},
sensitive=false,
comment=[l]{//},
morecomment=[s]{/*}{*/},
commentstyle=\color{OliveGreen}\ttfamily,
stringstyle=\color{OliveGreen}\ttfamily,
morestring=[b]',
morestring=[b]"
}
\usepackage{color}
\definecolor{gray97}{gray}{.97}
\definecolor{gray90}{gray}{.90}
\definecolor{gray75}{gray}{.75}
\definecolor{gray45}{gray}{.45}
\definecolor{codegreen}{rgb}{0,0.6,0}
\definecolor{codered}{rgb}{0.6,0,0}
\definecolor{codegray}{rgb}{0.5,0.5,0.5}
\definecolor{codepurple}{rgb}{0.58,0,0.82}
\lstset{ frame=single,
	framerule=0.2pt,
	framextopmargin=3pt,
	framexbottommargin=3pt,
	framexleftmargin=0.4cm,
	framesep=0.5pt,
	rulesep=0.5pt,
	backgroundcolor=\color{gray97},
	rulesepcolor=\color{black},
	xleftmargin=0.7cm,
	%
	stringstyle=\ttfamily,
	showstringspaces = false,
	basicstyle=\fontsize{6pt}{7pt}\ttfamily,
	keywordstyle=\color{magenta}\bfseries,
	numberstyle=\tiny\color{codegray},
	stringstyle=\color{codepurple},
	commentstyle=\color{codegreen},
	%
	numbers=left,
	numbersep=15pt,
	numberstyle=\tiny,
	numberfirstline = false,
	breaklines=true,
	escapeinside={(*@}{@*)},
	literate={~} {$\sim$}{1}
}

\lstdefinestyle{floating}{%
  frame=none,
  float=htb,
  captionpos=b
}

% context traits listings
\lstdefinestyle{ctxtraits}
 {language=JavaScript,
  frame=lines,
  showstringspaces=false,
  keywordstyle=\tt\bf,
  tabsize=3,
  style=floating,
  morekeywords={Trait, cop, Context, activate, deactivate, adapt, addObjectPolicy, manager}
}

%context traits environment    
\lstnewenvironment{ctxtraits}[1][]
 {\lstset{style=ctxtraits,#1}}{}  


 % Context Traits in line source code
\newcommand{\scode}[1]{\textrm{\texttt{#1}}}
\def\scode{\lstinline[style=ctxtraits]}

%----[ Commands ]---
%Latins
\newcommand{\eg}{\emph{e.g.,}\xspace}
\newcommand{\ie}{\emph{i.e.,}\xspace}
\newcommand{\etal}{\emph{et al.}\xspace}
\newcommand{\aka}{\emph{a.k.a.,}\xspace}
\newcommand{\cf}{\emph{cf.}\xspace}
\newcommand{\mref}{\textcolor{red}{[REF]}\xspace}
\newcommand{\secref}[1]{Section~\ref{#1}\xspace}
\newcommand{\figref}[1]{Fig.~\ref{#1}\xspace}
\newcommand{\listref}[1]{Listing~\ref{#1}\xspace}
\newcommand{\tabref}[1]{Table~\ref{#1}\xspace}

\newcommand{\itdroid}{\textsc{itDroid}\xspace}
\newcommand{\numapps}{\textcolor{red}{80}\xspace}


\newcommand{\ctx}[1]{\texttt{\textsc{#1}}}


%comments
% xcolor
\definecolor{author}{rgb}{.5, .5, .5}
\definecolor{comment}{rgb}{.1, .0, .9}
\definecolor{note}{rgb}{.9, .4, .0}
\definecolor{idea}{rgb}{.1, .7, .0}
\definecolor{missing}{rgb}{.9, .1, .0}
\definecolor{deleteme}{rgb}{.9, .1, .0}

\newcommand{\MARIO}[2][comment]{\authorcomment[#1]{MLV}{#2}}
\newcommand{\CAMILO}[2][comment]{\authorcomment[#1]{CEV}{#2}}
\newcommand{\ALEJO}[2][comment]{\authorcomment[#1]{AMR}{#2}}
\newcommand{\ANA}[2][comment]{\authorcomment[#1]{AMH}{#2}}
\newcommand{\MICHAEL}[2][comment]{\authorcomment[#1]{MOR}{#2}}
\newcommand{\LAURA}[2][comment]{\authorcomment[#1]{LBJ}{#2}}

\newcommand{\authorcomment}[3][comment]
  {\noindent
      \fbox{\footnotesize\textcolor{author}{\textsc{#2}}}
      \textcolor{#1}{\textsl{#3}}{}}



